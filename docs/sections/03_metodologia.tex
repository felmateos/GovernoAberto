\chapter{Metodologia}

Neste projeto, empregaremos uma metodologia que combina análise de conteúdo para avaliar a adesão das cidades selecionadas às Melhores Práticas para Dados na Web da W3C \cite{W3C}.

\section{Análise de Conteúdo}

Na análise de conteúdo, nos concentraremos nos portais de transparência de cada cidade para coletar e analisar os conjuntos de dados relacionados à educação. Esta etapa inclui o seguinte:

\begin{itemize}
    \item \textbf{Coleta de dados}: Visitaremos cada portal \cite{SP} \cite{CU} \cite{SA} \cite{MA} \cite{CG} e coletaremos os conjuntos de dados relevantes à educação disponíveis. Isso inclui dados sobre escolas, professores, alunos, desempenho dos alunos, financiamento e outros aspectos relevantes da educação.
    \item \textbf{Avaliação da qualidade dos dados}: Avaliaremos a qualidade dos dados coletados de acordo com os critérios estabelecidos pelas Melhores Práticas para Dados na Web da W3C \cite{W3CSUMMARY}. Isso inclui a análise da completude, precisão, relevância e consistência dos dados.
    \item \textbf{Avaliação da acessibilidade dos dados}: Avaliaremos a facilidade de acesso aos dados nos portais, a clareza das informações fornecidas e a presença de metadados que facilitam a compreensão e o uso dos dados.
\end{itemize}

\section{Comparação e Recomendações}

Após a análise de conteúdo, compararemos nossas descobertas entre as diferentes cidades. Isto nos permitirá identificar as cidades que estão liderando em termos de práticas de dados abertos e aquelas que podem precisar de melhorias.

Com base em nossas descobertas, formularíamos recomendações para cada cidade, sugerindo maneiras de melhorar a transparência, a qualidade e a acessibilidade dos dados de educação.

Os resultados desta análise permitirão não apenas uma avaliação aprofundada das práticas de dados abertos nos municípios selecionados, mas também fornecerão uma base para recomendações que possam promover uma maior transparência e acessibilidade dos dados de educação no futuro.

