\chapter{Objetivo}

O propósito principal deste projeto é avaliar a adesão às ''Melhores Práticas para Dados na Web'' da W3C \cite{W3C} nas cidades de São Paulo com foco específico em conjuntos de dados relacionados à educação disponíveis em seu respectivo portal de transparência.

\section{Objetivo Específico}

Através deste estudo, pretendemos:

\begin{itemize}
    \item \textbf{Analisar a qualidade e acessibilidade dos dados}: Avaliar se os conjuntos de dados educacionais nestas cidades estão em conformidade com as melhores práticas recomendadas pelo W3C \cite{W3CSUMMARY}, como a disponibilidade de descrições de alto nível dos conjuntos de dados, a granularidade dos dados, a existência de metadados apropriados e o uso de formatos abertos e amplamente utilizados.
    \item \textbf{Comparar as práticas entre diferentes cidades}: Comparar e contrastar a adesão a essas melhores práticas entre a cidade de São Paulo e outras capitais brasileiras, sendo estas: Campo Grande, Curitiba, Manaus e Salvador. O objetivo é metrificar, em termos de práticas de dados abertos, o desempenho de São Paulo através da comparação entre outros municípios.
    \item \textbf{Fornecer recomendações para aprimorar a abertura de dados}: Com base nas descobertas, oferecer sugestões para aprimorar a transparência e a abertura dos dados relacionados à educação em cada uma dessas cidades.
\end{itemize}

Para alcançar esses objetivos, este estudo utilizou alguns sites como recurso de apoio:

\begin{itemize}
    \item Portal de Transparência da Educação do Estado de São Paulo \cite{ENAPSP}
    \item QEdu \cite{QEDU}
    \item Pátio Digital da Prefeitura Municipal de São Paulo \cite{PATIODIGITAL}
    \item Censo Escolar do INEP \cite{EDUCACENSO}
    \item Portal da Transparência do Governo Federal – Seção Educação \cite{PORTALTRANSPARENCIA}
    \item Inep consulta de Escolas \cite{CATALOGOESCOLAS}
    \item Heuristic Evaluation. Usability Evaluation Materials. \cite{HE}
    \item Document Analysis as a Qualitative Research Method \cite{DA}
    \item Portal de Transparência - Curitiba \cite{CU}
    \item Portal de Transparência - Salvador \cite{SA}
    \item Portal da Transparência de Campo Grande \cite{CG}
    \item Portal da Transparência de Manuaus \cite{MA}
\end{itemize}

\section{Justificativa da Escolha dos Municípios}

A seleção dos municípios de São Paulo, Curitiba, Salvador, Manaus e Campo Grande para este estudo foi estrategicamente pensada considerando o seguinte:

\textbf{São Paulo} \cite{SP}: Como a maior cidade do Brasil, São Paulo possui o maior sistema educacional municipal do país. Além disso, a cidade é conhecida por seus esforços para promover a transparência e a inovação digital, tornando-a um caso interessante para análise.

\textbf{Curitiba} \cite{CU}: Curitiba é frequentemente referida como uma cidade inovadora no que se refere a soluções urbanas e também possui um histórico forte no que diz respeito à transparência e uso de dados abertos. Além disso, sua posição como uma das principais cidades da região Sul a torna uma adição valiosa ao estudo.

\textbf{Salvador} \cite{SA}: Como uma das maiores cidades do Nordeste e com uma complexa rede de ensino, Salvador oferece uma visão valiosa das práticas de dados abertos nesta região. Além disso, a cidade tem feito esforços significativos para melhorar a educação e a transparência, tornando-a uma escolha relevante para este projeto.

\textbf{Manaus} \cite{MA}: Manaus é a cidade mais populosa da região Norte e tem um contexto educacional único, devido a seus desafios geográficos e socioeconômicos. Analisar a abertura e transparência dos dados educacionais nesta cidade proporcionará uma perspectiva importante das práticas de dados abertos nesta região.

\textbf{Campo Grande} \cite{CG}: Representando a região Centro-Oeste, Campo Grande tem um sistema educacional significativo e é conhecida por seus esforços em relação à governança digital. A inclusão desta cidade fornecerá uma visão mais completa das práticas de dados abertos em todo o Brasil.

Cada cidade foi selecionada não apenas pela sua representatividade regional, mas também pelas suas características únicas em termos de tamanho, desafios educacionais e compromisso com a transparência e os dados abertos. Acreditamos que essa diversidade fortalecerá a análise e as conclusões do nosso estudo.

